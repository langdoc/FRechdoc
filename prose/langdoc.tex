 \documentclass[a4paper, 12pt, english]{article}
 \usepackage{polyglossia}
 \usepackage{babel}
 \usepackage[utf8]{inputenc}
 \usepackage{expex}
 \setromanfont{Charis SIL}
 
 \title{\LaTeX-example}
 \author{Niko Partanen}
 \date{7.4.2017}
 
\begin{document}
\maketitle 
This document demonstrates basic usage of \LaTeX, with especial focus in linguistic data. Actually the idea is to reproduce more or less the same document in various formats besides LaTeX, for example, in RMarkdown and Jupyter Notebook.

% This is a comment, it will not be visible in the output.

\ex
\begingl
\gla А сійӧ нинӧм эз шу//
\glb and he nothing neg.3\sc{sg.prs} say.\sc{conneg}//
\glft ‘And he doesn't say anything.’//
\endgl
\xe 
   
 \end{document}