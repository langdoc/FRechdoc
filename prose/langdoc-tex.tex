\documentclass[a4paper, 12pt, english]{article}

%%%Xelatex
\usepackage{xltxtra,xunicode}
%%%

%%%polyglossia
\usepackage{polyglossia}
\setdefaultlanguage{english}
%%%

%%%Expex
\usepackage{expex}
%%%

%%%Font
\setromanfont{Charis SIL}
%%%

%%%Biblatex
\usepackage[	
			language=english,
			backend=biber,
			style=authoryear,
%			maxcitenames=1,%threshold how many names in references in the text
%			maxbibnames=1,%threshold how many names in list of references
%			maxnames=1,%threshold how many names (makes to two former redundant)
			]{biblatex}
\bibliography{langdoc}
%%%
  
%%%Title
\title{\XeLaTeX-example}
\author{
	Niko Partanen
	\and
	Michael Rießler
	}
\date{\today}
%%%
 
\begin{document}

\maketitle 

This document demonstrates basic usage of \XeLaTeX, with special focus in linguistic data. Actually the idea is to reproduce more or less the same document in various formats besides LaTeX, for example, in RMarkdown and Jupyter Notebook.

% This is a comment, it will not be visible in the output.

\section{Linguistic examples}

\ex
\begingl
\gla А сійӧ нинӧм эз шу//
\glb and he nothing neg.3\sc{sg.pst} say.\sc{conneg}//
\glft ‘And he didn't say anything.’//
\endgl
\xe 

\section{Using BibLaTeX}

Here are a few basic examples for the use of citations with BibLaTeX. Similar result as with these functions can be reached in RMarkdown as well, but the control is far less nuanced.

\begin{itemize}
\item Mention of a whole work in the text:
	\subitem Possibilities of adding morphological analysis automatically to ELAN files were recently demonstrated by \cite{gerstenbergerEtAl2017b}.
\item Mention of an author with reference to a specific work:
	\subitem Possibilities of adding morphological analysis automatically to ELAN files were recently demonstrated by \textcite{gerstenbergerEtAl2017b}.
\item Mention of an author with reference to a specific work and specific pages:
	\subitem Possibilities of adding morphological analysis automatically to ELAN files were recently demonstrated by \textcite[10-11]{gerstenbergerEtAl2017b}.
\item Reference for a statement in the text:
	\subitem Komi is a Uralic language \parencite[2]{gerstenbergerEtAl2017b}.
\item Reference for a statement in the text, incl. additional notes:
	\subitem Komi is a Uralic language \parencite[against the statement made by][4]{gerstenbergerEtAl2017b}.
\item Mention of several authors with reference to specific works and specific pages:
	\subitem Possibilities of adding morphological analysis automatically to ELAN files were recently demonstrated by \textcites[10-11]{gerstenbergerEtAl2017b}[2]{gerstenbergerEtAl2017a}.
\end{itemize}
   
\printbibliography

\end{document}
